\documentclass[a4paper]{article}
\usepackage[top=1in, bottom=1.25in, left=1.25in, right=1.25in]{geometry}
\usepackage{amsmath}
\usepackage{multicol}
\usepackage{graphicx}
\RequirePackage{ltxcmds}[2010/12/07]
%opening
\title{Pulse Shaper}

\begin{document}

\maketitle

This blocks applies a time domain, finite impulse response filter to the signal. The filter's transfer function is defined by the vector \textit{impulseResponse}. It allows for passive filter mode operation via a boolean check.

\subsection*{Input Parameters}

\begin{itemize}
	\item filterType
	\item impulseResponseTimeLength
	\item rollOfFactor
	\item usePassiveFilterMode
\end{itemize}

\subsection*{Functional Description}

\subsection*{Input Signals}

\textbf{Number}: 1

\textbf{Type}: Sequence of Dirac Delta functions (ContinuousTimeDiscreteAmplitude)

\subsection*{Output Signals}

\textbf{Number}: 1

\textbf{Type}: Sequence of impulses modulated by the filter (ContinuousTimeContiousAmplitude)

\subsection*{Suggestions for future improvement}

Introduce other types of filters.

\end{document}