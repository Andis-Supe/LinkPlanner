\documentclass[a4paper]{article}
\usepackage[top=1in, bottom=1.25in, left=1.25in, right=1.25in]{geometry}
\usepackage{amsmath}
\usepackage{multicol}
\usepackage{graphicx}
\usepackage[utf8]{inputenc}
\usepackage[english]{babel}
\setlength{\parskip}{0.03cm plus4mm minus3mm}
\RequirePackage{ltxcmds}[2010/12/07]

\usepackage{hyperref}
%opening
\title{Decoder}

\begin{document}

\maketitle

This block accepts a complex electrical signal and outputs a sequence of binary values (0's and 1's). Each point of the input signal corresponds to a pair of bits.

\subsection*{Input Parameters}

\begin{itemize}
	\item\texttt{t\_integer} m\{ 4 \}
	\item vector$<$\texttt{t\_complex}$>$ iqAmplitudes\{ \{ 1.0, 1.0 \},\{ -1.0, 1.0 \},\{ -1.0, -1.0 \},\{ 1.0, -1.0 \} \};
\end{itemize}

\subsection*{Methods}
 
Decoder() {}
\bigbreak
Decoder(vector$<$Signal *$>$ \&InputSig, vector$<$Signal *$>$ \&OutputSig) :Block(InputSig, OutputSig) {}
\bigbreak
void initialize(void)
\bigbreak
bool runBlock(void)
\bigbreak
void setM(int mValue)
\bigbreak
void getM()
\bigbreak
void setIqAmplitudes(vector$<$\texttt{t\_iqValues}$>$ iqAmplitudesValues)
\bigbreak
vector$<$\texttt{t\_iqValues}$>$getIqAmplitudes()

\subsection*{Functional description}

This block makes the correspondence between a complex electrical signal and pair of binary values.

To do so it computes the distance in the complex plane between each value of the input signal and each value of the \textit{iqAmplitudes} vector selecting only the shortest one. It then converts the point in the IQ plane to a pair of bits making the correspondence between the input signal and a pair of bits.

NOTE

\pagebreak

\subsection*{Input Signals}

\subparagraph*{Number:} 1

\subparagraph*{Type:} Electrical complex (TimeContinuousAmplitudeContinuousReal)

\subsection*{Output Signals}

\subparagraph*{Number:} 1

\subparagraph*{Type:} Binary 

\subsection*{Examples}

As an example take an input signal with positive real and imaginary parts. It would correspond to the first point of the \textit{iqAmplitudes} vector and therefore it would be associated to the  pair of bits $00$. 

\subsection*{Sugestions for future improvement}


\end{document}