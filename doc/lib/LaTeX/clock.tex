\documentclass[a4paper]{article}
\usepackage[top=1in, bottom=1.25in, left=1.25in, right=1.25in]{geometry}
\usepackage{amsmath}
\usepackage{multicol}
\usepackage{graphicx}
\usepackage[utf8]{inputenc}
\usepackage[english]{babel}
\setlength{\parskip}{0.03cm plus4mm minus3mm}
\RequirePackage{ltxcmds}[2010/12/07]

\usepackage{hyperref}
%opening
\title{Clock}

\begin{document}

\maketitle

This block doesn't accept any input signal. It outputs one signal that corresponds to a sequence of Dirac's delta functions with a user defined \textit{period}.

\subsection*{Input Parameters}

\begin{itemize}
	\item period\{ 0.0 \};
	\item samplingPeriod\{ 0.0 \};
\end{itemize}

\subsection*{Methods}
 
Clock() {}
\bigbreak
Clock(vector$<$Signal *$>$ \&InputSig, vector$<$Signal *$>$ \&OutputSig) :Block(InputSig, OutputSig) {}
\bigbreak
void initialize(void)
\bigbreak
bool runBlock(void)
\bigbreak
void setClockPeriod(double per)
\bigbreak
void setSamplingPeriod(double sPeriod)
\subsection*{Functional description}


\pagebreak

\subsection*{Input Signals}

\subparagraph*{Number:} 0

\subsection*{Output Signals}

\subparagraph*{Number:} 1

\subparagraph*{Type:} Sequence of Dirac's delta functions. (TimeContinuousAmplitudeContinuousReal)

\subsection*{Examples} 

\subsection*{Sugestions for future improvement}


\end{document}