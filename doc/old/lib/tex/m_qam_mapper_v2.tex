\documentclass[../../sdf/tex/BPSK_system.tex]{subfiles}
\graphicspath{{../../images/}}
%opening
\onlyinsubfile{\title{M-QAM Mapper}}
\date{}

\begin{document}

\onlyinsubfile{\maketitle}
	

\subsection*{Input Parameters}

\begin{itemize}
	\item m 
	\item iqAmplitudes 
\end{itemize}

\subsection*{Functional Description}

This block does the mapping of the binary signal using a \textit{m}-QAM modulation. It atributes to each pair of bits a point in the I-Q space. The constellation is defined by the \textit{iqAmplitudes} vector.

\subsection*{Input Signals}

\textbf{Number}: 1

\textbf{Type}: Binary (DiscreteTimeDiscreteAmplitude)

\subsection*{Output Signals}

\textbf{Number}: 2

\textbf{Type}: Sequence of 1's and -1's (DiscreteTimeDiscreteAmplitude)

\subsection*{Examples}

\begin{figure}[h]
    --- MISSING IMAGE ---
	%\includegraphics[width=\textwidth]{MQAM2}
\end{figure}

\subsection*{Sugestions for future improvement}

\end{document}
