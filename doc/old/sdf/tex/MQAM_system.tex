\documentclass[a4paper]{article}
\usepackage[top=1in, bottom=1.25in, left=1.25in, right=1.25in]{geometry}
\usepackage{amsmath}
\usepackage{multicol}
\usepackage{graphicx}
\RequirePackage{ltxcmds}[2010/12/07]
%opening
\title{MQAM system}

\begin{document}

\maketitle

\section*{Binary Source}

This block generates a sequence of binary values (1 or 0) and it can work in four different modes: 

\begin{itemize}
	\item Random
	\item PseudoRandom 
	\item DeterministicCyclic 
	\item DeterministicAppendZeros 
\end{itemize}

\subsection*{Input Parameters}

\begin{multicols}{2}
	\begin{itemize}
		\item mode
		\item probabilityOfZero 
		\item patternLength 
		\item bitStream 
		\item numberOfBits 
		\item bitPeriod 
	\end{itemize}
\end{multicols}

\subsection*{Functional description}


\textbf{Random Mode}:
generates a 0 with probability \textit{probabilityOfZero} and a 1 with probability 1-\textit{probabilityOfZero}.

\textbf{Pseudorandom Mode}: 
generates a pseudorandom sequence with period $2^\textit{patternLength}-1$.

\textbf{DeterministicCyclic Mode}:
generates the sequence of 0's and 1's specified by \textit{bitStream} and then repeats it.

\textbf{DeterministicAppendZeros Mode}:
generates the sequence of 0's and 1's specified by \textit{bitStream} and then it fills the rest of the buffer space with zeros.

\subsection*{Input Signals}

\textbf{Number}: 0 or 1 (which would work as a trigger)

\textbf{Type}: Binary (DiscreteTimeDiscreteAmplitude)

\subsection*{Output Signals}

\textbf{Number}: 1 or more

\textbf{Type}: Binary (DiscreteTimeDiscreteAmplitude)

\subsection*{Examples}

\textbf{Random Mode}

\textbf{PseudoRandom Mode}

As an example consider a pseudorandom sequence with \textit{patternLength}=3 which contains a total of 7 ($2^3-1$) bits. In this sequence it is possible to find every combination of 0's and 1's that compose a 3 bit long subsequence with the exception of $000$. For this example the possible subsequences are $100$, $010$, $001$, $110$, $101$, $011$ and $111$. Some of these require wrap. 




\textbf{DeterministicCyclic Mode}

\textbf{DeterministicAppendZeros Mode}

%%%%%%%%%%%%%%%%%%%%%%%%%%%%%%%%%%%%%%%%%%%%%%%%%%%%%%%%%%%%%%%%%%%%%%%%%%%%%%%%%%%%%%%%%%%%%%%%%%%%%%%%%%%%%%%%%%%%%%%%%%%%%%%%%%%%%%%%%%%%%%%%%%%%%%%%%%%%%%%%%%%%%%%%%%%%%%%%%%%%%%%%%%%%%%%%%%%%%%%%%%%%%%%%%%%%%%%%%%%%%%%%%%%%%%%%%%%%%%%%%%%%%%%%%%%%%%%%%%%%%%%%%%%%%%%%%%%%%%%%%%%%

\section*{M-QAM mapper}

This block does the mapping of the binary signal using a \textit{m}-QAM modulation. It atributes to each pair of bits a point in the I-Q space. The constellation is defined by the \textit{iqAmplitudes} vector.

\subsection*{Input Parameters}

\begin{itemize}
	\item m 
	\item iqAmplitudes 
\end{itemize}

\subsection*{Functional Description}



\subsection*{Input Signals}

\textbf{Number}: 1

\textbf{Type}: Binary (DiscreteTimeDiscreteAmplitude)

\subsection*{Output Signals}

\textbf{Number}: 2

\textbf{Type}: Sequence of 1's and -1's (DiscreteTimeDiscreteAmplitude)

\subsection*{Examples}

\begin{figure}[h]
	\includegraphics[width=\textwidth]{MQAM2}
\end{figure}

\subsection*{Sugestions for future improvement}

\pagebreak

%%%%%%%%%%%%%%%%%%%%%%%%%%%%%%%%%%%%%%%%%%%%%%%%%%%%%%%%%%%%%%%%%%%%%%%%%%%%%%%%%%%%%%%%%%%%%%%%%%%%%%%%%%%%%%%%%%%%%%%%%%%%%%%%%%%%%%%%%%%%%%%%%%%%%%%%%%%%%%%%%%%%%%%%%%%%%%%%%%%%%%%%%%%%%%%%%%%%%%%%%%%%%%%%%%%%%%%%%%%%%%%%%%%%%%%%%%%%%%%%%%%%%%%%%%%%%%%%%%%%%%%%%%%%%%%%%%%%%%%%%%%%

\section*{Discrete to Continuous Time}

This block converts a signal from a discrete time signal to a continuous time signal. To do so it reads the input signal buffer value, puts it in the output signal buffer and it fills the rest of the space available for thar symbol with zeros.

\subsection*{Input Parameters}

\begin{itemize}
	\item numberOfSamplesPerSymbol 
\end{itemize}

\subsection*{Functional Description}

\subsection*{Input Signals}

\textbf{Number}: 1

\textbf{Type}: Sequence of 1's and -1's. (DiscreteTimeDiscreteAmplitude)

\subsection*{Output Signals}

\textbf{Number}: 2

\textbf{Type}: Sequence of Dirac Delta functions (ContinuousTimeDiscreteAmplitude)

\subsection*{Example}

\begin{figure}[h]
	\includegraphics[width=\textwidth]{MQAM4}
\end{figure}

\subsection*{Sugestions for future improvement}

\pagebreak

%%%%%%%%%%%%%%%%%%%%%%%%%%%%%%%%%%%%%%%%%%%%%%%%%%%%%%%%%%%%%%%%%%%%%%%%%%%%%%%%%%%%%%%%%%%%%%%%%%%%%%%%%%%%%%%%%%%%%%%%%%%%%%%%%%%%%%%%%%%%%%%%%%%%%%%%%%%%%%%%%%%%%%%%%%%%%%%%%%%%%%%%%%%%%%%%%%%%%%%%%%%%%%%%%%%%%%%%%%%%%%%%%%%%%%%%%%%%%%%%%%%%%%%%%%%%%%%%%%%%%%%%%%%%%%%%%%%%%%%%%%%%

\section*{Pulse Shapper}

This blocks applies a raised-cosine filter to the signal. The filter's transfer function is defined by the vector \textit{impulseResponse}.

\subsection*{Input Parameters}

\begin{itemize}
	\item filterType
	\item impulseResponseTimeLength
	\item rollOfFactor
\end{itemize}

\subsection*{Functional Description}

\subsection*{Input Signals}

\textbf{Number}: 1

\textbf{Type}: Sequence of Dirac Delta functions (ContinuousTimeDiscreteAmplitude)

\subsection*{Output Signals}

\textbf{Number}: 1

\textbf{Type}: Sequence of impulses modulated by the filter (ContinuousTimeContiousAmplitude)

\subsection*{Example}

\begin{figure}[h]
	\includegraphics[width=\textwidth]{MQAM6}
\end{figure}

\subsection*{Sugestions for future improvement}

Introduce other types of filters.

\pagebreak

%%%%%%%%%%%%%%%%%%%%%%%%%%%%%%%%%%%%%%%%%%%%%%%%%%%%%%%%%%%%%%%%%%%%%%%%%%%%%%%%%%%%%%%%%%%%%%%%%%%%%%%%%%%%%%%%%%%%%%%%%%%%%%%%%%%%%%%%%%%%%%%%%%%%%%%%%%%%%%%%%%%%%%%%%%%%%%%%%%%%%%%%%%%%%%%%%%%%%%%%%%%%%%%%%%%%%%%%%%%%%%%%%%%%%%%%%%%%%%%%%%%%%%%%%%%%%%%%%%%%%%%%%%%%%%%%%%%%%%%%%%%%

\section*{IQ Modulator}

This blocks takes the two input signals that correspond to the part of the signal in phase and in quadrature and produces a complex signal, that contains information about the amplitude and phase. 

\subsection*{Input Parameters}

\begin{itemize}
	\item outputOpticalPower
	\item outputOpticalWavelength
	\item outputOpticalFrequency
\end{itemize}

\subsection*{Functional Description}

The complex signal is multiplied by $\frac{1}{2}\sqrt{\textit{outputOpticalPower}}$ in order to reintroduce the information about the energy (or power) of the signal. This information was omitted ...

\subsection*{Input Signals}

\textbf{Number}: 2

\textbf{Type}: Sequence of impulses modulated by the filter (ContinuousTimeContiousAmplitude))

\subsection*{Output Signals}

\textbf{Number}: 1

\textbf{Type}: Complex signal (ContinuousTimeContiousAmplitude)

\subsection*{Example}
\begin{figure}[h]
	\includegraphics[width=\textwidth]{MQAM8}
\end{figure}

\subsection*{Sugestions for future improvement}

\end{document}
