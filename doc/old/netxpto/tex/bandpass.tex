\documentclass{article}
%\usepackage[portuguese]{babel}
\usepackage[utf8x]{inputenc}
\usepackage{amsmath}
\usepackage{float}
\usepackage{graphicx}
\usepackage{multicol}
\usepackage{multirow}
\usepackage[margin=1in]{geometry}
\usepackage{indentfirst}
\usepackage{amsfonts}

\begin{document}


\title{Band-pass signal definition}
\maketitle

The signal power in band-pass representation can be defined in multiple ways. Let us first define the initial signal $S(t)$ and assuming this has a real amplitude $A(t)$:

\begin{equation}
S(t)=A(t)\cos(\omega t+\theta(t))=\frac{A(t)}{2}\left(e^{i(\omega t+\theta(t))}+e^{-i(\omega t+\theta(t))}\right).
\end{equation}
%
We can now define the band-pass representation signal as:

\begin{equation}
s(t)=a(t)\cos(\theta(t)) 
\end{equation}

For simplicity, and with no loss of generality, let us assume that $a(t)=a$ and $A(t)=A$, that is, both are time constant. Let us also assume that $\theta(t)=0$. Using the definition for instant power we get the relation between $a$ and $A$. 

\begin{equation}
P=\lim_{T \to 0}\frac{1}{T}\int_0^Tdt~S(t)=\frac{A^2}{2}=\left(\frac{A}{\sqrt{2}}\right)^2\Rightarrow a(t)=\frac{A(t)}{\sqrt{2}}
\end{equation}

\end{document}
