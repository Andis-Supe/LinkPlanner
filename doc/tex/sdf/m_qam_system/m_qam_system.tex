\clearpage
\section{M-QAM Transmission System}

\begin{tcolorbox}	
\begin{tabular}{p{2.75cm} p{0.2cm} p{10.5cm}} 	
\textbf{Student Name}  &:& ana Luisa\\
\textbf{Starting Date} &:& September 22, 2017\\
\textbf{Goal}          &:& Study of a M-QAM transmission system.
\end{tabular}
\end{tcolorbox}

Oblivious Transfer (OT) is a fundamental primitive in multi-party computation. The one-out-of-two OT consists in a communication protocol between Alice and Bob. At the beginning of the protocol Alice has two messages $m_1$ and $m_2$ and Bob wants to know one of them, $m_b$, without Alice knowing which one, i.e. without Alice knowing $b$, and Alice wants to keep the other message private, i.e. without Bob knowing $m_{\bar{b}}$. therefore two conditions must be fulfilled:
\begin{enumerate}
	\item{The protocol must be concealing, i.e at the beginning of the protocol Bob does not know nothing about the messages know by Alice, while at the end of the protocol Bob will learn the message $b$ chose by him.}
	\item{The protocol is oblivious, i.e Alice cannot learn anything about the choice of Bob, i.e. about bit $b$, and Bob cannot learning nothing about the other message $m_{\bar{b}}$.}
\end {enumerate}

\subsection{OT Protocol Detailed}


