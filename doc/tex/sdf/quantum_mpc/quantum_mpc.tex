\section{Quantum Multi-Party Computation}

\begin{refsection}

\subsection{Our Approach}
\subsubsection{1 out of 2 Oblivious Transfer}
This section will provide the description of a 1-2 Oblivious Transfer (OT) based on the appliance of a Quantum Oblivious Key Distribution Protocol (QOKD). The 1-2 OT consists in a two party, Alice and Bob, communication protocol. Supposing that Alice has two messages $\{m_1,m_0\}$ length $s$. Bob wants to know one of those in such a way that:
\begin{itemize}
\item Alice doesn't know Bob's choice, i.e. the protocol is oblivious;
\item Bob doesn't get any information on $m_\bar{b}$, the protocol is concealing.
\end{itemize}
Considering the notation of a canonical quantum oblivious transfer protocol, let $U=\{+,\times\}^n\times\{0,1\}^n$, where $+, \times$ stand for the rectilinear and diagonal bases - physically the rectilinear or diagonal polarization of the photon. The general algorithm of the protocol can be described as:
 \begin{itemize}
 	\item Step 1:\\
 	Alice picks a random uniformly chosen $\big(a,g\big)\,\in\,U$, and sends Bob photons $i$, $1\leq i \leq n$ with polarizations given by the bases $a\big[i\big]$ and states $g\big[i\big]$.
 	\item Step 2:\\
 	Bob picks a random uniformly chosen $b\in\{+,\times\}^n$, measures photons $i$ in basis $b\left[i\right]$ and records the results, if a photon is detected, as $h\left[i\right]\in\{0,1\}$. Bob then makes a quantum commit of all $n$ pairs $\left(b\left[i\right],h\left[i\right]\right)$ to Alice.
 	\item Step 3:\\
 	Alice picks a random uniformly picks a random uniformly chosen subset $R\subset\{1,2,...,n\}$ and tests the commitment made by Bob at positions in $R$. If more $\delta n$ (acceptance threshold) positions $i\in R$ reveal $a\left[i\right]=b\left[i\right]$ and $g\left[i\right]\neq h\left[i\right]$ then Alice stops protocol; otherwise, the test result is accepted.
 	\item Step 4:\\
 	Alice announces the base $a$. Let $T_0$ be the set of $1\leq i \leq n$ such that $a\left[i\right]=b\left[i\right]$ and let $T_1$ be the set of all $1\leq i \leq n$ such that $a\left[i\right]\ne b\left[i\right]$. Bob chooses $I_0,I_1\subset T_0-R, T_1-R$ and sends $S_i=\{I_\bar{i},I_i\}$, wishing to know $m_i$, $i\in\{0,1\}$.
 	\item Step 5:\\
 	Alice defines two encryption keys $K_0,K_1$ in such a way that $K_i=g\big[I_i\big]$ for $i=0 \vee i=1$. Alice then cyphers both messages: $m_{\mathtt{coded}}=\{m_0\oplus K_0,m_1\ \oplus K_1\}$ and sends the result $m$ to Bob.
 	\item Step 6:\\
 	Bob will then decode $m$ using the values of his initially chosen basis:  $b\left[S_i\right]$ with $i\in\{0,1\}$, according to his preference. $m_{\mathtt{decoded}}=m_{\mathtt{coded}}\oplus b\left[S_i\right]$. The output of this process will be $m_{\mathtt{decoded}}$ that will have the correct message in the first or last $s^{th}$ positions if he chose $m_0$ or $m_1$, respectively.
 \end{itemize}
To formalize and proof the security of this system one has to think of the proceedings of a hypothetical dishonest Bob and a third party eavesdropper, named Eve.
\begin{itemize}
\item Step 1:\\
Dishonest Bob has no advantage in being dishonest at this point.
\item Step 2:\\
Dishonest Eve transfers some information from this pulse into her quantum system and she uses that information to modify the residual state of the pulse which is sent to Bob.
\item Step 3:\\
Dishonest Bob executes a coherent measurement on the pulse received in order to determine:
- whether or not he declares this pulse as detected;\\
- the bit that he commits to Alice.
\item Step 4:\\
Having learnt Alice's string of basis $a\left[i\right]$ Bob chooses this (...)
\end{itemize}

% bibliographic references for the section ----------------------------
\clearpage
\printbibliography[heading=subbibliography]
\end{refsection}
\addcontentsline{toc}{subsection}{Bibliography}
\cleardoublepage
% --------------------------------------------------------------------- 