\clearpage

\section{White Noise}

\begin{tcolorbox}	
	\begin{tabular}{p{2.75cm} p{0.2cm} p{10.5cm}} 	
		\textbf{Header File}   &:& white\_noise\_20180118.h \\
		\textbf{Source File}   &:& white\_noise\_20180118.cpp \\
	\end{tabular}
\end{tcolorbox}

\maketitle
This block generates a gaussian pseudo-random noise signal with a given spectral density. It can be initialized with three different seeding methods to allow control over correlation and reproducibility:

\begin{enumerate}
	\item DefaultDeterministic
	\item RandomDevice
	\item Selected
\end{enumerate}

This block does not accept any input signal. It produces can produce a real or complex output, depending on the used output signal.

\subsection*{Input Parameters}

\begin{table}[h]
	\centering
	\begin{tabular}{|c|c|c|c|c}
		\cline{1-4}
		\textbf{Parameter} & \textbf{Type} &\textbf{Values} &   \textbf{Default}& \\ \cline{1-4}
		seedType 		 & enum & DefaultDeterministic, RandomDevice, Selected & RandomDevice \\ \cline{1-4}
		spectralDensity  & real & > 0  			& $10^{-4}$ \\ \cline{1-4}
		seed 	   		 & int & $\in$ [1, $2^{32}-1$] 	& 1 \\ \cline{1-4} \cline{1-4}
	\end{tabular}
	\caption{White noise input parameters}
	\label{table:noise_in_par}
\end{table}

\subsection*{Methods}

WhiteNoise(vector<Signal *> \&InputSig, vector<Signal *> \&OutputSig) :Block(InputSig, OutputSig){};
\bigbreak	
void initialize(void);
\bigbreak
bool runBlock(void);
\bigbreak
void setNoiseSpectralDensity(double SpectralDensity) { spectralDensity = SpectralDensity; }
\bigbreak
double const getNoiseSpectralDensity(void){ return spectralDensity; }
\bigbreak
void setSeedType(SeedType sType){ seedType = sType; };
\bigbreak
SeedType const getSeedType(void){ return seedType; };
\bigbreak
void setSeed(int newSeed) { seed = newSeed; }
\bigbreak
int getSeed(void){ return seed; }

\subsection*{Functional description}

The \textit{seedType} parameter allows the user to select between one of the three seeding methods to initialize the pseudo-random number generators (PRNGs) responsible for generating the noise signal.


\textbf{DefaultDeterministic}: Uses default seeds to initialize the PRNGs. These are different for all generators used within the same block, but remain the same for sequential runs or different \textit{white\_noise} blocks. Therefore, if more than one \textit{white\_noise} block is used, another seeding method should be chosen to avoid producing the exact same noise signal in all sources.

\textbf{RandomDevice}: Uses randomly chosen seeds using \textit{std::random\_device} to initialize the PRNGs.

\textbf{SingleSelected}: Uses one user selected seed to initialize the PRNGs. The selected seed is passed through the variable \textit{seed}. If more than one generator is used, additional seeds are created by choosing the next sequential integers. For instance, if the user selected seed is $10$, and all the four PRNGs are used, the used seeds will be $[10, 11, 12, 13]$.

The noise is then obtained from a gaussian distribution with variance equal to the spectral density. If the signal is complex, the noise is calculated independently for the real and imaginary parts, with half the spectral density in each.


%\subparagraph*{AllSelected}
%Uses user selected seeds to initialize the PRNGs. The selected seed is passed through the variable \textit{seed}. If more than one generator is used, additional seeds are created by choosing the next sequential integers.

\subsection*{Input Signals}


\subparagraph*{Number:} 0

\subsection*{Output Signals}

\subparagraph*{Number:} 1 or more

\subparagraph*{Type:} RealValue, ComplexValue or ComplexValueXY

\subsection*{Examples}

\paragraph*{Random Mode}

\subsection*{Suggestions for future improvement}
