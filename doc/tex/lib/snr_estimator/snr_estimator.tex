\clearpage

\section{SNR Estimator}

\begin{tcolorbox}	
	\begin{tabular}{p{2.75cm} p{0.2cm} p{10.5cm}} 	
		\textbf{Header File}   &:& snr\_estimator.h \\
		\textbf{Source File}   &:& snr\_estimator.cpp \\
		\textbf{Version}	   &:& 20180313 (Responsible: Andoni Santos)
	\end{tabular}
\end{tcolorbox}

\subsection*{Input Parameters}

\begin{table}[H]
	\centering
	\begin{tabular}{|l|l|l|}
		\hline
		\textbf{Name}  & \textbf{Type}  & \textbf{Value}    \\ \hline
		Confidence     & double         & 0.95              \\ \hline
		MidReportSize  & integer        & 0                 \\ \hline
%		LowestMinorant & double         & $1\times10^{-10}$ \\ \hline
	\end{tabular}
\end{table}

\subsection*{Input Signals}

\textbf{Number}: 1\\
\textbf{Type}: OpticalSignal or TimeContinuousAmplitudeContinuousReal

\subsection*{Output Signals}

\textbf{Number}: 1\\
\textbf{Type}: Binary (TimeDiscreteAmplitudeContinuousReal)

\subsection*{Functional Description}

This block accepts one OpticalSignal or TimeContinuousAmplitudeContinousReal signal, estimates the signal-to-noise ratio and outputs the estimated value. It also outputs a \textit{.txt} file reporting the estimated signal-to-noise ratio, a count of the number of measurements and the corresponding bounds for a given confidence level.

\subsection*{Theoretical Description}\label{bercalc}
The $\widehat{\text{BER}}$ is obtained by counting both the total number received bits, $N_T$, and the number of coincidences, $K$, and calculating their relative ratio:
\begin{equation}
\widehat{\text{BER}}=1-\frac{K}{N_T}.
\end{equation}
The upper and lower bounds, $\text{BER}_\text{UB}$ and $\text{BER}_\text{LB}$ respectively, are calculated using the Clopper-Pearson confidence interval, which returns the following simplified expression for $N_T>40$~\cite{almeida2016continuous}:
\begin{align}
	\text{BER}_\text{UB}&=\widehat{\text{BER}}+\frac{1}{\sqrt{N_T}}z_{\alpha/2}\sqrt{\widehat{\text{BER}}(1-\widehat{\text{BER}})}+\frac{1}{3N_T}\left[2\left(\frac{1}{2}-\widehat{\text{BER}}\right)z_{\alpha/2}^2+(2-\widehat{\text{BER}})\right]\\
	\text{BER}_\text{LB}&=\widehat{\text{BER}}-\frac{1}{\sqrt{N_T}}z_{\alpha/2}\sqrt{\widehat{\text{BER}}(1-\widehat{\text{BER}})}+\frac{1}{3N_T}\left[2\left(\frac{1}{2}-\widehat{\text{BER}}\right)z_{\alpha/2}^2-(1+\widehat{\text{BER}})\right],
\end{align}
where $z_{\alpha/2}$ is the $100\left(1-\frac{\alpha}{2}\right)$th percentile of a standard normal distribution.





\bibliographystyle{unsrt}

\bibliography{bibliography} 