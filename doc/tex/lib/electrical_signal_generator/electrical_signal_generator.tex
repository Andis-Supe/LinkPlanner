\clearpage

\section{Electrical Signal Generator}

\maketitle
This block generates time continuous amplitude continuous signal, having only one output and no input signal.

\subsection{ContinuousWave}
Continuous Wave the function of the desired signal. This must be introduce by using the function \textit{setFunction(ContinuousWave)}. This function generates a continuous signal with value 1. However, this value can be multiplied by a specific gain, which can be set by using the function \textit{setGain()}. This way, this block outputs a continuous signal with value $1 \times \textrm{gain}$.

\subsection*{Input Parameters}

	\begin{itemize}
		\item ElectricalSignalFunction signalFunction{}\linebreak
		(ContinuousWave)
		\item samplingPeriod\{\}\linebreak
        (double)
		\item symbolPeriod\{\} \linebreak
        (double)
		
	\end{itemize}

\subsection*{Methods}

ElectricalSignalGenerator() \{\};
\bigbreak	
void initialize(void);
\bigbreak	
bool runBlock(void);
\bigbreak	
void setFunction(ElectricalSignalFunction fun)
ElectricalSignalFunction getFunction()
\bigbreak	
void setSamplingPeriod(double speriod)
double getSamplingPeriod()
\bigbreak
void setSymbolPeriod(double speriod)
double getSymbolPeriod()

\bigbreak	
void setGain(double gvalue) 
double getGain() 


\subsection*{Functional description}

The \textit{signalFunction} parameter allows the user to select the signal function that the user wants to output.

\subparagraph*{Continuous Wave}
Outputs a time continuous amplitude continuous signal with amplitude 1 multiplied by the gain inserted.


\subsection*{Input Signals}

\subparagraph*{Number:} 0

\subparagraph*{Type:}No type

\subsection*{Output Signals}

\subparagraph*{Number:} 1 

\subparagraph*{Type:} TimeContinuousAmplitudeContinuous

\subsection*{Examples}


\subsection*{Sugestions for future improvement}

Implement other functions according to the needs.

