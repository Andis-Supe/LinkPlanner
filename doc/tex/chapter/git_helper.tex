% ------------------------------------------------------------------------
\chapter{Git Helper}

Git creates and maintains a database that store versions of a repository, i.e. versions of a folder.
To create this database for a specific folder the Git application must be installed on the computer. Open the Git console program and go to the specific folder and execute the following command:
\\[0mm]
\par\textbf{git init}
\\[5mm]
The Git database is created and stored in the folder \emph{.git} in the root of your repository. The Git commands allow you to manipulate this database.

\section{Data Model}

To understand Git is fundamental to understand the Git data model.
\\[5mm]
Git stores the following objects:

\begin{itemize}
    \item[\textbullet] {commits - text files that store a description of the repository;}
    \item[\textbullet] {trees - text files that store a description of a folder;}
    \item[\textbullet] {blobs - the files that exist in your repository;}
\end{itemize}

\noindent The objects are stored in the folder \emph{.git/objects}. Each store object is identified by its SHA1 hash message, i.e. 20 bytes which identifies unequivocally the object. Note that 20 bytes can be represented by a 40 characters hexadecimal string. The ID of each object is the 40 characters hexadecimal string. Note that Git is creating a new object every time a file or a folder is changed. The Git database store all committed versions of a file.

A commit object is identified by a SHA1 message, and has the following information: a pointer for tree (the root of your repository), a pointer for the previous commit, the author of the commit, the committer and a commit message.\\
%
Example of a commit file contend:\\
\\
tree 2c04e4bad1e2bcc0223239e65c0e6e822bba4f16\\
parent bd3c8f6fed39a601c29c5d101789aaa1dab0f3cd\\
author NetXPTO <netxpto@gmail.com> 1514997058 +0000\\
committer NetXPTO <netxpto@gmail.com> 1514997058 +0000\\
\\
2018-01-03, after meeting Daniel\\

A tree object is identified by a SHA1 message, and has a list of blobs and trees that are inside that tree.\\
\\
Example of a tree file contend:\\
\\
\begin{tabular}{l l l l}
100644 & blob & bdb0cabc87cf50106df6e15097dff816c8c3eb34 &   .gitattributes\\
100644 & blob & 50492188dc6e12112a42de3e691246dafdad645b &   .gitignore\\
100644 & blob & 8f564c4b3e95add1a43e839de8adbfd1ceccf811 &   bfg-1.12.16.jar\\
040000 & tree & de44b36d96548240d98cb946298f94901b5f5a05 &   doc\\
040000 & tree & 8b7147dbfdc026c78fee129d9075b0f6b17893be &   garbage\\
040000 & tree & bdfcd8ef2786ee5f0f188fc04d9b2c24d00d2e92 &   include\\
040000 & tree & 040373bd71b8fe2fe08c3a154cada841b3e411fb &   lib\\
040000 & tree & 7a5fce17545e55d2faa3fc3ab36e75ed47d7bc02 &   msbuild\\
040000 & tree & b86efba0767e0fac1a23373aaf95884a47c495c5 &   mtools\\
040000 & tree & 1f981ea3a52bccf1cb00d7cb6dfdc687f33242ea &   references\\
040000 & tree & 86d462afd7485038cc916b62d7cbfc2a41e8cf47 &   sdf\\
040000 & tree & 13bfce10b78764b24c1e3dfbd0b10bc6c35f2f7b &   things\_to\_do\\
040000 & tree & 232612b8a5338ea71ab6a583d477d41f17ebae32 &  visualizerXPTO\\
040000 & tree & 1e5ee96669358032a4a960513d5f5635c7a23a90 &   work\_in\_progress\\
\end{tabular}
\\
A blob is identified by a SHA1 message, and has the file contend compressed.\\

\section{Database Folders and Files}

\subsection{Objects Folder}

Git stores the database and the associated information in a set of folders and files inside the the folder \emph{.git} in the root of your repository.

The folder \emph{.git/objects} stores information about all objects (commits, trees and blobs).
The objects are stored in files inside folders.
The name of the folders are the 2 first characters of the SHA1 40 characters hexadecimal string.
The name of the files are the other 38 hexadecimal characters of the SHA1.
The information is compressed to save same space but it can be access using some applications.

\subsection{Refs Folder}

The \emph{refs} folder has the following folders \emph{refs/heads},.

\subsubsection{Heads Folder}
The \emph{refs/heads} folder has information about all branches.
Each branch is associated with a text file, the name of the file is the name of the branch.
The file has inside the hash of the commit pointed by that branch.


\section{Commands}

\textbf{git branch}
git branch --set-upstream-to=<remote>/<branch> Develop.Nelson

\textbf{git cat-file}
git cat-file -t <hash>

shows the type of the objects

git cat-file -p <hash>

shows the contend of the object

\textbf{git clone}

\textbf{git diff}

show the changes with relation to the previous commit.

\textbf{git fecth -all}

\textbf{git init}

\textbf{git log}

shows a list of commits from reverse time order (goes back on time), i.e. shows the history of your repository. The history of a repository can be represented by a directed acyclic graph (dag for short), pointing in the forward direction in time.

options

--graph

 shows a graphical representation of the repository commits history.

\textbf{git show}

Shows what is new in the last commit.

\textbf{git status}

\section{The Configuration Files}

There is a config file for each repository that is stored in the \emph{.git/} folder with the name \emph{config}.\\
\\
There is a config file for each user that is stored in the \emph{c:/users/<user name>/} folder with the name \emph{.gitconfig}.\\
\\
To open the \emph{c:/users/<user name>/.gitconfig} file type:\\
\\
\textbf{git config --global -e}

\section{Applications}

\subsection{Meld}

%# --------------------------------------D I F F -----------------------------
%
%[diff]
%    guitool = meld
%
%[difftool "meld"]
%    cmd = \"C:/Program Files (x86)/Meld/Meld.exe\" \"$LOCAL\" \"$REMOTE\" --label \"DIFF (ORIGINAL MY)\"
%	
%# --------------------------------------M E R G E -----------------------------
%
%[merge]
%    tool = meld
%
%[mergetool "meld"]
%    cmd = \"C:/Program Files (x86)/Meld/Meld.exe\" --auto-merge \"$LOCAL\" \"$BASE\" \"$REMOTE\" --output \"$MERGED\" --label \"MERGE (REMOTE BASE MY)\"
%    trustExitCode = false
%
%[mergetool]
%    # don't ask if we want to skip merge
%    prompt = false
%
%    # don't create backup *.orig files
%    keepBackup = false

\subsection{GitKraken}

\section{Error Messages}

\subsection{Large files detected}

Clean the repository with the \href{https://rtyley.github.io/bfg-repo-cleaner}{BFG Repo-Cleaner}.\\
\\
Run the Java program:\\

java -jar bfg-1.12.16.jar \texttt{-{}-}strip-blobs-bigger-than 100M\\
\\
This program is going to remote from your repository all files larger than 100MBytes. After do:\\

git push \texttt{-{}-}force.
